\documentclass[10pt,a4paper]{article}
\usepackage[utf8]{inputenc}
\usepackage[francais]{babel}
\usepackage{amsmath}
\usepackage{amsfonts}
\usepackage{amssymb}
\usepackage{graphicx}
\usepackage{lipsum}

\author{Antoine Audouin}

\newcommand{\MONTITRE}{Fonctionnement d’un réseau}
\newcommand{\MONSOUSTITRE}{Fiche 6 (Nat/Pat, Redirection)}
\title{
\begin{tabular}{p{3.5cm} r}
& {\Huge {\bf \MONTITRE}}\\
& {\huge \MONSOUSTITRE}\\
& version du \today{}
\end{tabular}
}

\begin{document}
\maketitle
\newpage
\tableofcontents
\newpage

\section{CAHIER DES CHARGES}
\begin{itemize}
\item Seuls le serveur Web et le serveur de messagerie seront accessibles depuis l’Internet, sans restriction particulière
\bigbreak
\item La nécessité d’une politique de sécurité différente pour ces serveurs a conduit au découpage du réseau en deux zones et donc deux réseaux logiques distincts.
\bigbreak
\item La zone accessible à partir d’Internet est généralement appelée DMZ pour « Zone démilitarisée » (autrement dit plus accessible ou moins protégée).
\bigbreak
\item La zone plus protégée (le réseau local ou LAN) est séparée de la DMZ par un routeur qui sera également capable de filtrer.
\bigbreak
\item Chaque zone est dotée de son propre switch ici, mais certaines entreprises mettent en place des VLANs (réseaux virtuels) sur un même équipement pour isoler les flux.
\bigbreak
\item Le réseau de la DMZ est en 172.16.0.0 (adresse privée) tandis que le réseau LAN est en 192.168.1.0 (adresse privée également).
\bigbreak
\item Dans la DMZ, le serveur WEB écoute sur le port « bien connu » 80 du protocole de transport TCP. Le serveur de messagerie écoute lui sur le port 25 du protocole TCP.
\bigbreak
\item Un serveur est également installé dans le réseau local et écoute sur le port 5000.
\bigbreak
\item Le routeur contrôlera les accès aux deux réseaux et aux différents serveurs, c’est pourquoi il dispose de 3 interfaces actives : une sur le LAN, une autre sur le DMZ, la troisième (carte d’accès distant) sur Internet.
\bigbreak
\item Le réseau de l’entreprise « PARTNER » est un réseau public (200.100.40.0) mais qui souhaite masquer ses adresses lorsqu’il communique sur Internet.
\bigbreak
\item Le dernier réseau est un réseau complètement indépendant, qui nous permettra de réaliser des tests. L’entreprise « 7-EXTRA » qui possède de réseau utilise un adressage privé en 10.0.0.0.
\bigbreak
\item Il se trouve que les entreprises « PARTNER » et « 7-EXTRA » ont le même fournisseur d’accès, mais cela n’a aucune incidence particulière.
\bigbreak
\item L’entreprise « PARTNER » a donc des relations privilégiées avec l’entreprise « LANIMAN » : elle doit pouvoir accéder au serveur présent dans son réseau local.	
NB : Ce serveur est nommé « Srv LANO » sur le schéma.
\bigbreak
\item Le réseau de « 7-EXTRA », en revanche, ne devra accéder qu’aux serveurs de la DMZ, comme n’importe quelle autre entreprise étrangère aux activités de « LANIMAN ». Les serveurs de la DMZ ont tous une adresse en 172.16.21.X.

\end{itemize}
\subsubsection{Mise en place du réseau physique}
\begin{itemize}
\item Respecter strictement l’agencement « géographique » pour faciliter le débogage du professeur en cas de besoin.
\bigbreak
\item Placer un routeur type ADSL (avec une carte d’accès distant reliée à Internet via un câble télécom) pour chacun des trois sites : LANIMAN, PARTNER, 7-EXTRA.
\bigbreak
\item Placer un switch pour chaque réseau IP et le relier au routeur adéquat. Sur le site de LANIMAN, les deux switchs (LAN et DMZ) seront connectés au même routeur.
\bigbreak
\item Placer deux postes sur chaque réseau logique (réseau IP)
\bigbreak
\item Nommer les actifs (routeurs et switchs) comme indiqué ci-dessous pour faciliter le repérage : « sw XXXX » et « RT-XXXX »
\bigbreak
\item Nommer les postes (serveurs et stations) comme ci-dessous également : « Srv XXX » pour les serveurs, « M. XXXX » ou « Mme XXX » pour les stations de travail.
\end{itemize}
\bigbreak
\center
\includegraphics[width=13cm]{../Pictures/Image_latex/Schema_maquette_NAT.png} 
\endcenter
\newpage
\subsubsection{Mise en place de la configuration IP}

\begin{itemize}
\item Configurer pour chaque poste l’adresse de passerelle adéquate lui permettant l’accès à Internet.
\bigbreak
\item Tout poste doit disposer d’une adresse de passerelle
\bigbreak
\item Tout routeur connecté à Internet doit avoir une route par défaut permettant de joindre les postes reliés Internet, et donc ceux connectés à l’autre FAI (puisqu’Internet est simulé par 2 FAI).
\end{itemize}
\bigbreak
\textbf{NOTE :} 
\bigbreak
\begin{itemize}
\item RT-FW est directement connecté à 3 réseaux (deux réseaux privés + Internet). Il faut lui rajouter une route par défaut utilisant comme passerelle le point de connexion à Internet de son FAI.
\newline
 NB : Le point de connexion n’est rien d’autre qu’un autre routeur en fait, mais appartenant au FAI.
\bigbreak
\item RT-DIST est directement connecté à 2 réseaux. Il faut de la même manière lui ajouter une route par défaut
\bigbreak
\item RT-EXT est directement connecté à 2 réseaux (un réseau privé et Internet). Il faut de la même manière lui ajouter une route par défaut
\end{itemize}
\newpage
\section{Accès à Internet pour les postes du réseau 10.0.0.0}
\begin{itemize}
\bigbreak
\item Ajouter un poste que vous nommerez « www.gogol »
\bigbreak
\item Ajouter à ce poste une carte d’accès distant et la relier directement à Internet sur le dernier point libre, via un câble Télécom.
\bigbreak
\item Installer un serveur web sur ce poste (il faut passer en mode Application pour ajouter un logiciel serveur sur un poste, ici HTTP)
\end{itemize}

\subsection{Questions}
\bigbreak
Le message parvient-il à www.gogol ?

\bigbreak
\textbf{Le message ne parvient pas à www.gogol. Le message d'erreur est le suivant :}
\bigbreak
\includegraphics[width=7cm]{../Pictures/Image_latex/Message_erreur.png} 
\bigbreak
La réponse part-elle de www.gogol ?
\medbreak
\textbf{La réponse provient du routeur, qui possède la MAC 27.}
\medbreak
\includegraphics[width=15cm]{../Pictures/Image_latex/reponse_routeur.PNG} 
\bigbreak
Parvient-elle au poste de Mme Extin ?
\medbreak
\textbf{\lipsum}
\bigbreak
Si la réponse ne part pas du serveur HTTP, c’est que vous avez oublié d’indiquer une passerelle par défaut pour ce serveur.
\medbreak
\textit{La passerelle par défaut c’est le routeur qui lui donne accès au « reste du monde », à savoir le routeur de son FAI qui a pour adresse : 172.11.0.2}
\bigbreak
Retentons une communication après configuration de la passerelle :
\medbreak
La réponse part-elle de www.gogol ?


\end{document}
